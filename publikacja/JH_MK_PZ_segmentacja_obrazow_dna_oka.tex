\documentclass[conference]{IEEEtran}
\IEEEoverridecommandlockouts
% The preceding line is only needed to identify funding in the first footnote. If that is unneeded, please comment it out.
\usepackage{cite}
\usepackage{amsmath,amssymb,amsfonts}
\usepackage{algorithmic}
\usepackage{graphicx}
\usepackage{textcomp}
\usepackage{xcolor}
\usepackage{polski}
\usepackage[utf8]{inputenc}
\def\BibTeX{{\rm B\kern-.05em{\sc i\kern-.025em b}\kern-.08em
    T\kern-.1667em\lower.7ex\hbox{E}\kern-.125emX}}
    
\usepackage[hidelinks]{hyperref}
\begin{document}

\title{Metody segmentacji naczyń krwionośnych na obrazach dna oka}

\author{
\IEEEauthorblockN{Jakub Harasymowicz}
\IEEEauthorblockA{
@student.pwr.edu.pl}
\and
\IEEEauthorblockN{Maciej Kinal}
\IEEEauthorblockA{
@student.pwr.edu.pl}
\and
\IEEEauthorblockN{Patryk Ząbkiewicz}
\IEEEauthorblockA{
221394@student.pwr.edu.pl}
}

\maketitle

\begin{abstract}
W~niniejszej pracy zostaje podjęta próba przetworzenia obrazów uzyskanych podczas badania dna oka, a~dokładniej mówiąc -- ich segmentacji za pomocą różnych metod. Ma to na celu wyodrębnienie naczyń krwionośnych, których nienaturalna ilość i~rozmiar może wskazywać na chorobę zwaną \textit{retinopatią cukrzycową}. Posegmentowane obrazy dna oka uwidaczniające naczynia krwionośne mogą stanowić jedną z~cech, która może zostać wykorzystana w~algorytmach uczenia maszynowego do automatyzacji detekcji retinopatii cukrzycowej, która z kolei stanowi globalny problem zdrowotny.
\end{abstract}

\begin{IEEEkeywords}
fundus image processing, fundus image segmentation, diabetic retinopathy
\end{IEEEkeywords}

\section{Wprowadzenie}
Jednym z~powikłań trwającej długi okres czasu cukrzycy może być negatywnie oddziaływanie na małe naczynia krwionośne w~siatkówce oka, prowadzące do choroby zwanej \textit{retinopatią cukrzycową}. Choroba ta przebiega w~kilku etapach zaawanasowania, powodując różne dolegliwości wzroku, w~najgorszych przypadkach wywołując nawet całkowitą ślepotę. Dotyczy to około 2,4 miliona przypadków na świecie, stawiając tym samym retinopatie na 5 miejscu, jeśli chodzi o~przyczyny utraty wzroku u~osób w~wieku 25-75 lat \cite{b1}.

Jedną z technik obrazowania, która umożliwia śledzenie rozwoju retinopatii jest obrazowanie dna oka (oftalmoskopia, fundoskopia). Obraz dna oka pozwala uwidocznić takie kliniczne zmiany jak mikrotętniaki (drobne czerwone okrągłe plamki), wysięki (nieregularne żółte plamki), wylewy do ciała szklistego. W~najpóźniejszym stadium retinopatii cukrzycowej -- retinopatii proliferacyjnej -- może objawiać się także w~zamykaniu naczyń oraz proliferacji, czyli powstawaniu nowych naczyń w~znaczącej ilości \cite{b2}.

Jako że duża ilość drobnych naczyń krwionośnych może wskazywać na retinopatię cukrzycową, informację tę można wykorzystać w~automatyzacji rozpoznawania tej choroby. W~tym celu, w~niniejszej pracy zostanie przeprowadzone przetworzenie obrazów -- segmentacja -- mające na celu wyodrębnienie naczyń krwionośnych z~obrazów dna oka.

\section{Baza obrazów}
\section{Metody}

\subsection{Segmentacja progowa}
\subsection{Segmentacja za pomocą skupisk}

\begin{thebibliography}{00}
\bibitem{b1} R. Lee, T.Y. Wong, C. Sabanayagam, ,,Epidemiology of diabetic retinopathy, diabetic macular edema and related vision loss'', Eye and Vision, vol. 2 pp. 17, Wrzesień 2015.
\bibitem{b2} M.I. Lopez-Galvez, F. Manco Lavado, J.C. Pastor, ,,Diabetic Retinopathy: An Overview'', Handbook of Nutrition, Diet and the Eye, pp. 41--51, 2014.
\end{thebibliography}

\end{document}
